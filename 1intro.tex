% !TEX root = 0main.tex
\section{Introduction}
\label{sec:intro}

The collaborative nature of open source software development has led to the advent of social coding platforms centred around the git version control system, such as \gh, BitBucket, and GitLab.
These platforms bring the collaborative nature and code reuse of open source software development to another level, via facilities like forking, pull requests, cherry-picking, \ldots.
Developers may fork a \textit{mainline repository} into a new \textit{forked repository} and take governance over the latter while preserving the full revision history.
Before the advent of social coding platforms, forking was rare and was typically intended to compete with the original project~\cite{Linus:2012Perspectives,Gregorio:2012,Viseur:2012Forks,Linus:2013CodeForking,Linus:2011ToFork,Gamalielsson:2014Sustainability}.

With the rise of the pull-based development model~\cite{Gousios:2014ICSE}, forking is much more frequent than before and the community typically makes a distinction between forks depending on their purpose~\cite{Zhou:2020}.
\textit{Social forks} are created for isolated development with the goal of contributing back to the mainline.
On the other hand, \textit{variant forks} are created by splitting off a new development branch, not necessarily with the intention of contributing back to the mainline.
These variant forks allow to steer the development into another direction, while leveraging the mainline project that defines or adheres to some standards~\cite{sung:ICSE:2020}.

Many studies have already investigated the motivations behind variant forks in the context of open source projects~\cite{Linus:2012Perspectives,Gregorio:2012,Viseur:2012Forks,Linus:2013CodeForking,Linus:2011ToFork,Gamalielsson:2014Sustainability}.
However, most of them have been conducted before the rise of social coding platforms and it is known that \gh has significantly changed the perception and practices of forking~\cite{Zhou:2020}.
Developers frequently perceive social forks as alternatives to the original mainline project, and variant forks often evolve out of them rather than being planned deliberately.
Variant forks in the \gh era allows different projects to peacefully co-exist rather than compete, so in that sense it is worthwhile to revisit the motivation for creating variant forks (\textit{why?}).

Social coding platforms in the \gh era offer many facilities for code sharing (i.e., pull requests, cherry-picking, \ldots).
So if projects co-exist, one would expect that variant forks take advantage of this common ancestry, and frequently exchange code snippets with one another.
Despite advanced code-sharing facilities, Businge et al. observed very limited code integration between the mainline and the variant projects~\cite{businge:emse:2021}.
This suggests that the code sharing facilities in themselves are not enough for graceful co-evolution, so in that sense it is worthwhile to investigate impediments for co-evolution (\textit{how?}).

\sd{Be careful: I massaged the RQ slightly. If you copy them in the experimental set-up use the commands I created for them.}

\newcommand*{\RQOne} [1] {Why do developers create and maintain variants on \gh?}
\newcommand*{\RQTwo} [1] {How do variant projects evolve with respect to the mainline?}


\noindent
Consequently, we define the following research questions:
\begin{itemize}
\item $RQ_1$ --- \textit{\RQOne}
The literature pre-dating git reported three categories of motivations for creating variant forks: technical (e.g., diverging features), governance (e.g., diverging interests), legal (e.g., diverging licences), or personal (e.g., diverging principles)?
With this research question, we aim to investigate if the motivations for variant forks reported in literature still hold or whether there are new factors coming into play?

\item $RQ_2$ --- \textit{\RQTwo}
\git offers advanced code-sharing facilities, yet at least one paper reported very limited code integration between the mainline and the variant projects~\cite{businge:emse:2021}
With this research question we investigate the overlap between the teams maintaining the mainlines and variant forks and how they communicate with one another.
As such we hope to identify impediments for co-evolution.
\end{itemize}


The investigations are based on an online survey we conducted with 112 maintainers involved in 105 active variant forks hosted on \gh.
%We performed an exploratory investigation of maintainers of variants hosted on \gh. We conducted an online survey that was answered by 112 maintainers of 105 active variant projects.
%
Our contributions are manifold:
we identify a new reasons for creating and maintaining variant forks;
we identify and categorize different code reuse and change propagation practices between a variant and its mainline;
we confirm that little code integration occurs between a variant and its mainline, and uncover concrete reasons for this phenomenon;
and finally, we discuss the implications of these findings and how tools can help to achieve an efficient code integration and collaboration between mainlines and diverging variant forks.

\sd{Need to squeeze this so it fits on a single page}

% \nd \textbf{We report a number of findings}: 1) Our study identifies a number of common fine-grained motivations for creating and maintaining variants that include: different goals\,/content\,/communities, customization, supporting personal projects, supporting the upstream, localization, up-taking a frozen feature.
% 2) Our study has identified a number of categories relating to different reuse practices of variants that include: integrating only bug\,/security fixes, updates only on specific features, all updates excluding specific features, and only updates that pass CI. For example, we have discovered software reuse of a family of applications from the ``cryptocurrency world'', which is dedicated project software ecosystem~\cite{tommens:2020}.
% 3) Our study extends the previous study by providing concrete reasons relating to reasons the little code integration observed between mainline and variants. For example, technically diverged variants, such as, those targeting different goals, those implementing different technologies, and those maintaining a specific feature of a mainline that was abandoned (a frozen feature). Based on these categories of variants, we have also discussed implications to tool building can help aid efficient code reuse between the mainlines and diverged variants.

